%\PassOptionsToPackage{gray}{xcolor}
\documentclass[hyperref={pdfpagelabels=false},12pt]{beamer}
\setbeamertemplate{frametitle}[default][center]
\mode<presentation>
{
 \usetheme{default}      % or try Darmstadt, Madrid, Warsaw, ...
 \usecolortheme{default} % or try albatross, beaver, crane, ...
 \usefonttheme{default}  % or try serif, structurebold, ...
 \setbeamertemplate{navigation symbols}{}
 \setbeamertemplate{caption}[numbered]
} 


\usepackage[utf8]{inputenc}
%\usepackage{multirow}
%\usepackage{wasysym}
%\usepackage{upquote}
\usepackage{listings}
\usepackage{gensymb}
\usepackage{array}
\usepackage{times}
\usepackage{xcolor}
\usepackage{default}
\usepackage{ulem}

%\usetheme{Pittsburgh}

\xdefinecolor{darkgreen}{rgb}{0.11,0.64,0.22}
\title[Basic Unix/Linux]{{Basic Unix/Linux}}
\author[Basic Unix/Linux]{{Kim, Barry, Ketan}}
\date{}

\beamertemplatenavigationsymbolsempty

\begin{document}
\begin{frame}[plain]
\titlepage
\end{frame}

\begin{frame}
\frametitle{Getting Started}
Know your machine!
\end{frame}

\begin{frame}
\frametitle{How it all clicked!}
Brief History.
\end{frame}

\begin{frame}
\frametitle{Understanding the Unix Command}
Locate commands; internal vs. external.
\end{frame}

\begin{frame}
\frametitle{General-purpose utilities}
date, who, uname, echo, passwd, bc, man, time
\end{frame}

\begin{frame}
\frametitle{Navigate the Filesystem}
cd, pwd, mkdir, find
\end{frame}

\begin{frame}
\frametitle{Handle Files}
less, more, cat, ls, chmod, chown, umask
\end{frame}

\begin{frame}
\frametitle{The Shell}
\end{frame}

\begin{frame}
\frametitle{The Environment}
aliases, ps, 
\end{frame}

\begin{frame}
\frametitle{Simple Filters}
head, tail, sort, uniq, tr, cut, paste 
\end{frame}

\begin{frame}
\frametitle{More filters}
awk, sed, grep
\end{frame}

\begin{frame}
\frametitle{The Shell}
wildcards, escaping and quoting, pipe and redirection, tee, variables
\end{frame}

\begin{frame}
\frametitle{Remote Connectivity and Networks}
ssh, scp, rsync, ftp, curl
\end{frame}

\begin{frame}
\frametitle{Exercises}
\end{frame}

\end{document}


